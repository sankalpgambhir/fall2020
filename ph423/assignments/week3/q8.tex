\question
\sankalp{I got this one.}

\begin{alphaparts}

\questionpart
Writing matrices for the Hamiltonians $\Hamiltonian_0$ and $\Hamiltonian'$ with
$\statepsi_a$ and $\statepsi_b$ as basis

\begin{align}
    \Hamiltonian_0 &= \begin{pmatrix}
        E_a & 0 \\ 
        0   & E_b
    \end{pmatrix}, \\
    \Hamiltonian' &= \begin{pmatrix}
        E_a & h \\ 
        h   & E_b
    \end{pmatrix}~, \textnormal{and} \\
    \Hamiltonian &= \Hamiltonian_0 + \Hamiltonian' \nonumber \\
                 &= \begin{pmatrix}
                    2E_a & h \\ 
                    h   & 2E_b
                 \end{pmatrix}~.
\end{align}

The eigenvalues for this combined Hamiltonian are easily calculated by
considering the eigenvalue equation and taking a determinant to get
$\determinant(E \cdot \Identity - \Hamiltonian) = 0.$ The eigenvalues $E$ which
solve this equation are thus given by 

\begin{align}
    (E - 2E_a) \cdot (E - 2E_b) - h^2 = 0 \\
    E = (E_a + E_b) \pm \sqrt{(E_a + E_b)^2 + h^2}  
\end{align}

We can also obtain the eigenvectors $(c_1, c_2)$ from the same equation,
obtaining easily the relation

\begin{equation}
    \frac{c_1}{c_2} = \frac{(E_b - E_a) \pm \sqrt{(E_b - E_a)^2 + h^2} }{h}
\end{equation}

We note that the two summands on the right side can be split using the familiar
trigonometric identity between the tangent and secant to get

\begin{equation}
    \frac{c_1}{c_2} = \tan \alpha \pm \sec \alpha
\end{equation}

\questionpart
To obtain an estimate of the ground state energy, we assume the trial
wavefunction of the form $\statepsi = \cos{\theta} \cdot \statephi_a +
\sin{theta} \cdot \statephi_b$. Notably, the state is assumed to be normalized.
Hence, the energy eigenvalues are obtained simply by calculating the expectation
value of the Hamiltonian $\Hamiltonian$ and minimizing it with respect to the
parameter $\theta$

\begin{align}
    \expectationvalue{\Hamiltonian}{\statepsi}  &=
    \begin{pmatrix}
        \cos{\theta} & \sin{\theta}
    \end{pmatrix}
    \begin{pmatrix}
        2E_a & h \\ 
        h   & 2E_b
    \end{pmatrix} \begin{pmatrix}
        \cos{\theta} \\ \sin{\theta}
    \end{pmatrix} \\
    &= 2E_a \cos^2{\theta} + 2E_b \sin^2{\theta} + 2h \cos{\theta}\sin{\theta}
\end{align}

and minimizing

\begin{align}
    \dv{\theta} \expectationvalue{\Hamiltonian}{\statepsi}  &= 0 \nonumber \\
    \dv{\theta} (2E_a \cos^2{\theta} + 2E_b \sin^2{\theta} + 2h \cos{\theta}\sin{\theta})  &= 0 \nonumber \\
    -4E_a \cos{\theta}\sin{\theta} + 4E_b \sin{\theta}\cos{\theta} 
    + 2h \cos^2{\theta} - 2h \sin^2{\theta}  &= 0~.
\end{align}

Observing that this can be converted to an equation in $\cos {2\theta}$ and
$\sin 2\theta$, and finally dividing by $\cos {2\theta}$ we get an equation
solely in the tangent

\begin{equation}
    \tan 2\theta = \frac{h}{E_a - E_b}
\end{equation}

Recognizing this as the tangent substitution we performed in part a, we get 

\begin{equation}
    \tan 2\theta = - \cot \alpha~.
\end{equation}

Using the tangent double-angle formula and solving the resultant quadratic
equation, we obtain the ratio

\begin{equation}
    \tan \theta = \frac{\sin \theta}{\cos \theta} = \tan \alpha \pm \sec \alpha
\end{equation}

which are precisely the eigenvectors we got in part a! So the expectation values
corresponding to these are given by the eigenvalues themselves.

The variational principle is accurate in this case because the trial
wavefunction spans the entire state space, so minimizing over it, we guarantee a
global minima.

\end{alphaparts}