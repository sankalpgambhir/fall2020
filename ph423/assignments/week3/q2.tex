    \parth{doing question 2. Someone go through afterwards}

    \question*{Calculate the variational bound on the ground state energy of the Hydrogen atom using the trial wavefunction
    \begin{equation*}
        \psi_{\alpha}(r) = 
        \begin{cases}
            \left( 1 - \frac{r}{\alpha}\right) & r \leqslant \alpha, \\
            0 & r \geqslant \alpha 
        \end{cases}
    \end{equation*}
    How is $\alpha_{min}$ related to the Bohr radius?
    }

    The Hamiltonian for an electron in the Hydrogen atom is 

    \begin{equation}
        \label{eqn:hydrogenhamiltonian}
        \hat{H} = -\frac{\hbar^2}{2\mu r^2} \left[\frac{\partial}{\partial r}\left(r^2\frac{\partial}{\partial r}\right) + \frac{1}{\sin \theta}\frac{\partial}{\partial\theta}\left(\sin\theta \frac{\partial}{\partial \theta}\right) + \frac{1}{\sin^2\theta}\frac{\partial^2}{\partial \phi^2} \right] - \frac{e^2}{4\pi\varepsilon_0}\frac{1}{r}
    \end{equation}

    obtained from 

    \begin{equation*}
        \hat H = -\frac{\hbar^2}{2\mu}\nabla^2 + P(\vec r) = -\frac{\hbar^2}{2\mu}\nabla^2 - \frac{Ze^2}{4\pi\varepsilon_0 r}
    \end{equation*}

    and then expressing the Laplacian in spherical coordinates and setting Z = 1. Here $\mu$ is the reduced mass of the system.
    
    The only undetermined parameter in our trial wavefunction is $\alpha$. We'll have to minimize the expectation value of the Hamiltonian of an electron in a Hydrogen atom (\ref{eqn:hydrogenhamiltonian}) with respect to this undetermined parameter by solving 

    \begin{equation}
        \label{eqn:variational}
        \frac{d\langle H \rangle(\alpha)}{d\alpha} = 0
    \end{equation}

    So the first thing to do is to solve for $\langle H \rangle$, and then minimize this value with respect to $\alpha$

    \begin{equation*}
        \langle H \rangle = \frac{\bra{\psi}H\ket{\psi}}{\braket{\psi}{\psi}}
    \end{equation*}

    in wave function formalism, this takes the form of an integral over all space.

    \begin{equation}
        \label{eqn:expectationvalue}
        \langle H \rangle = \frac{\iiint_{all space} (r^{2}sin(\theta))\psi^{*}(r,\theta,\phi)\hat{H}\psi(r,\theta,\phi) \,dr\,d\theta\,d\phi}{\iiint_{all space} (r^{2}sin(\theta))\psi^{*}(r,\theta,\phi)\psi(r,\theta,\phi) \,dr\,d\theta\,d\phi}
    \end{equation}

    Our trial wave function, as given in the question, is real-valued, and spherically symmetric. \\
    We first calculate the norm of the wave function. Our trial wave function is
    \begin{equation*}
        \psi_{\alpha}(r) = 
        \begin{cases}
            \left( 1 - \frac{r}{\alpha}\right) & r \leqslant \alpha, \\
            0 & r \geqslant \alpha 
        \end{cases}
    \end{equation*}

    \begin{equation*}
    \begin{split}
        therefore \braket{\psi}{\psi} & = \iiint_{all space} (r^{2}sin(\theta))\psi^{*}(r,\theta,\phi)\psi(r,\theta,\phi) \,dr\,d\theta\,d\phi \\
        & = 4\pi \int_{0}^{\infty} r^{2} \psi^{2}(r)\,dr \\
        & = 4\pi \int_{0}^{\alpha} r^{2}\left( 1 - \frac{r}{\alpha}\right)^{2}\,dr \\
        & = 4\pi \int_{0}^{\alpha} \left( r^2 - \frac{2r^3}{\alpha} + \frac{r^4}{\alpha^2} \right)\,dr \\
        & = 4\pi \frac{\alpha^3}{30}
    \end{split}
    \end{equation*}
    
    Due to the symmetries in the system, the numerator in \ref{eqn:expectationvalue} takes the form

    \begin{equation*}
        \bra{\psi}H\ket{\psi} = 4\pi \int_{0}^{\infty} r^{2} \psi(r)\hat{H}\psi(r)\,dr
    \end{equation*}

    We need to find out what the action on the Hamiltonian operator on our trial wavefunction is.

    \begin{equation*}
    \begin{split}
        \hat{H}\psi(r) & = -\frac{\hbar^2}{2\mu    r^2} \left[\frac{\partial}{\partial r}\left(r^2\frac{\partial\psi(r)}{\partial r}\right)\right] - \frac{e^2}{4\pi\varepsilon_0}\frac{\psi(r)}{r} \\
        & = -\frac{\hbar^2}{2\mu   r^2} \left[\frac{\partial}{\partial r}\left(\frac{-r^{2}}{\alpha}\right)\right] - \frac{e^2}{4\pi\varepsilon_0}\left(\frac{1}{r} - \frac{1}{\alpha}\right) \\
        & = -\frac{\hbar^2}{2\mu   r^2} \left[ \frac{-2r}{\alpha}\right] - \frac{e^2}{4\pi\varepsilon_0}\left(\frac{1}{r} - \frac{1}{\alpha}\right) \\
        & = \frac{\hbar^2}{\mu r\alpha} - \frac{e^2}{4\pi\varepsilon_0}\left(\frac{1}{r} - \frac{1}{\alpha}\right)
    \end{split}
    \end{equation*}

    The above derivation for $\hat{H}\psi(r)$ is for $r \leqslant \alpha$. For $r \geqslant \alpha$, our wavefunction is zero, and this $\hat{H}\psi(r)$ also evaluates to zero.\\
    Thus, for $\bra{\psi}H\ket{\psi}$, we have

    \begin{equation*}
    \begin{split}
        \bra{\psi}H\ket{\psi} & = 4\pi \int_{0}^{\infty} r^{2} \psi(r)\hat{H}\psi(r)\,dr \\
        & = 4\pi \int_{0}^{\alpha} r^{2}\left(1 - \frac{r}{\alpha}\right)\left[\frac{\hbar^2}{\mu  r\alpha} - \frac{e^2}{4\pi\varepsilon_0}\left(\frac{1}{r} - \frac{1}{\alpha}\right)\right]\,dr \\
        & = 4\pi \int_{0}^{\alpha} \left(1 - \frac{r}{\alpha}\right)\left[\frac{\hbar^{2}r}{\mu    \alpha} - \frac{e^2}{4\pi\varepsilon_0}\left(r - \frac{r^{2}}{\alpha}\right)\right]\,dr \\
        & = 4\pi \int_{0}^{\alpha} \left[\left(\frac{\hbar^{2}}{\mu    \alpha} - \frac{e^{2}}{4\pi\varepsilon_0}\right)r + \left( -\frac{\hbar^{2}}{\mu   \alpha^{2}} + \frac{e^{2}}{2\pi\varepsilon_{0}\alpha} \right)r^{2} - \left( \frac{e^{2}}{4\pi\varepsilon_{0}\alpha^{2}}\right)r^{3}\right]\,dr \\
        & = 4\pi \left[\left(\frac{\hbar^{2}}{\mu  \alpha} - \frac{e^{2}}{4\pi\varepsilon_0}\right)\frac{\alpha^{2}}{2} + \left( -\frac{\hbar^{2}}{\mu  \alpha^{2}} + \frac{e^{2}}{2\pi\varepsilon_{0}\alpha} \right)\frac{\alpha^3}{3} - \left( \frac{e^{2}}{4\pi\varepsilon_{0}\alpha^{2}}\right)\frac{\alpha^4}{4}\right] \\
        & = 4\pi \left[ \frac{\hbar^{2}}{\mu   }\left( \frac{\alpha}{2} - \frac{\alpha}{3}\right) + \frac{e^{2}}{4\pi\varepsilon_0}\left( -\frac{\alpha^2}{2} + \frac{2\alpha^2}{3} - \frac{\alpha^2}{4}\right) \right] \\
        & = 4\pi \left( \frac{\hbar^{2}\alpha}{6\mu} - \frac{\alpha^{2}e^2}{48\pi\varepsilon_{0}} \right)
    \end{split}
    \end{equation*}

    Using this result, and our result for the norm of $\psi$, we have

    \begin{equation*}
    \begin{split}
        \langle H \rangle & = \frac{4\pi \left( \frac{\hbar^{2}\alpha}{6\mu} - \frac{\alpha^{2}e^2}{48\pi\varepsilon_{0}} \right)}{4\pi \frac{\alpha^3}{30}} \\
        & = 30 \left( \frac{\hbar^{2}}{6\mu\alpha^2} - \frac{e^2}{48\pi\varepsilon_{0}\alpha} \right)
    \end{split}
    \end{equation*}

    Substituting this in \ref{eqn:variational}, we have

    \begin{equation*}
        \frac{d\langle H \rangle(\alpha)}{d\alpha} = 0 
    \end{equation*}

    \begin{equation*}
    \begin{split}
        \therefore & \frac{d}{d\alpha} \left( \frac{\hbar^{2}}{6\mu\alpha^2} - \frac{e^2}{48\pi\varepsilon_{0}\alpha} \right) = 0 \\
        \implies & \left( -\frac{\hbar^{2}}{3\mu\alpha_{min}^{3}} + \frac{e^2}{48\pi\varepsilon_{0}\alpha_{min}^{2}}\right) = 0 \\
        \implies & \alpha_{min} = \frac{16\pi\varepsilon_{0}\hbar^{2}}{\mu e^2}
    \end{split}
    \end{equation*}

    The variational bound on the ground state energy is given by

    \begin{equation*}
    \begin{split}
        \langle H \rangle_{min} & = 30 \left( \frac{\hbar^{2}}{6\mu\alpha_{min}^{2}} - \frac{e^2}{48\pi\varepsilon_{0}\alpha_{min}} \right) \\
        & = 30 \left( \frac{\mu e^4}{6*256\pi^{2}\varepsilon_{0}^{2}\hbar^{2}} - \frac{\mu e^4}{4*256\pi^{2}\varepsilon_{0}^{2}\hbar^{2}} \right) \\
        & = 30\left( -\frac{\mu e^4}{12*256\pi^{2}\varepsilon_{0}^{2}\hbar^{2}} \right) \\
        & =  -\frac{5\mu e^4}{3*256\pi^{2}\varepsilon_{0}^{2}\hbar^{2}}
    \end{split}
    \end{equation*}

    This value of $\alpha_{min}$ that we obtained is 4 times the reduced Bohr radius.

    \begin{equation*}
    \begin{split}
        & \alpha_{min} = \frac{16\pi\varepsilon_{0}\hbar^{2}}{\mu e^2} \\
        & a_{0}^{*} = \frac{4\pi\varepsilon_{0}\hbar^{2}}{\mu e^2} \\
    \end{split}
    \end{equation*}
    \begin{equation}
        \therefore \alpha_{min} = 4a_{0}^{*}
    \end{equation}

