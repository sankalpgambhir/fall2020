\question

\parth{Doing question 8, might have to check later}
\sankalp{No, me BRUH}

    The original Hamiltonian is of the form $H_0 = AL^2 + BL_z$. There's a small perturbation added to this Hamiltonian of the form $H_1 = CL_y$, where C is a constant. Thus, we have

    \begin{gather*}
        H_0 = AL^2 + BL_z \\
        L_y = \left( \frac{L_+ - L_-}{2i} \right)
    \end{gather*}

    The eigenkets of $H_0$ are the simultaneous eigenkets of $L^2$ and $L_z$, $\ket{l,m}$. The energy eigenvalues here are $E_{l,m} = A\hbar^2l(l+1) + B\hbar m$.

    $L_y = \left( \frac{L_+ - L_-}{2i} \right)$, on the other hand, is off-diagonal in the $\ket{l,m}$ basis (owing to the ladder operators seen). Due to this, the first order energy shift $\bra{l,m}CL_y\ket{l,m}$ vanishes, and we are left with the second order energy shifts

    \begin{equation*}
        \begin{split}
            E^{'}_{l,m} & = E_{l,m}^{(0)} + C^2\sum_{m^{'} \neq m} \frac{|\bra{l,m^{'}}\left( \frac{L_+ - L_-}{2i} \right)\ket{l,m}|^2}{E_{l,m}^{(0)} - E_{l,m^{'}}^{(0)}} + O(C^3) \\
            & = E_{l,m}^{(0)} + (C\hbar)^2\frac{(l-m)(l + m + 1)}{-4B\hbar} + (C\hbar)^2\frac{(l+m)(l - m + 1)}{4B\hbar} + O(C^3) \\
            & = E_{l,m}^{(0)} + \hbar m \frac{C^2}{2B} + O(C^3)
        \end{split}
    \end{equation*}

    Similarly, the shifts in the eigenfunctions are
    \begin{equation*}
        \begin{split}
           \ket{l,m} & = \ket{l,m}_0 + C\sum_{m' \neq m} \frac{\bra{l,m'}\left( \frac{L_+ - L_-}{2i} \right)\ket{l,m}}{E_{l,m}^{(0)} - E_{l,m^{'}}^{(0)}} \ket{l,m^{'}} \\
           & = \ket{l,m}_0 + \frac{C}{2i}\sum_{m' \neq m} \frac{\bra{l,m'}\left( L_+ - L_- \right)\ket{l,m}}{E_{l,m}^{(0)} - E_{l,m^{'}}^{(0)}} \ket{l,m^{'}}\\
           & = \ket{l,m}_0 + \frac{C\hbar}{2i}\sum_{m' \neq m} \frac{\sqrt{l(l+1) - m(m+1)}\bra{l,m'}\ket{l,m+1} - \sqrt{l(l+1) - m(m-1)}\bra{l,m'}\ket{l,m-1}}{E_{l,m}^{(0)} - E_{l,m^{'}}^{(0)}} \ket{l,m^{'}}
        \end{split}
    \end{equation*}


    We make these calculations for the case $l = 1$. Thus, the corrections in each state corresponding to $m = -1, 0, +1$ are -

    \begin{gather*}
        E^{'}_{1,-1} = E_{1,-1} - \frac{C^2\hbar}{2B} = 2A\hbar^2 - \frac{(C^2 + 2B^2)\hbar}{2B}\\
        E^{'}_{1,0} = E_{1,0} = 2A\hbar^2\\
        E^{'}_{1,1} = E_{1,1} + \frac{C^2\hbar}{2B} = 2A\hbar^2 + \frac{(C^2 + 2B^2)\hbar}{2B}
    \end{gather*}

    \begin{gather*}
        \ket{1,-1}^{'} = \ket{1,-1} - \frac{C}{\sqrt{2}Bi}\ket{1,0} \\
        \ket{1,0}^{'} = \ket{1,0} - \frac{C}{\sqrt{2}Bi}(\ket{1,1} + \ket{1,-1})\\
        \ket{1,1}^{'} = \ket{1,1} - \frac{C}{\sqrt{2}Bi}\ket{1,0} 
    \end{gather*}

Solving the system exactly, we recognize that the new states must be withing the
linear span of the $\ket{l, m}$ states. We then write the effect of the
Hamiltonian on an arbitrary such state to obtain the eigenvalue condition.

\begin{align}
    H \sum_{l, m}a_{l, m}\ket{l, m} &= (A \LAngularmomentum^2 + B\LAngularmomentum_z + C\LAngularmomentum_y) \sum_{l, m}a_{l, m}\ket{l, m}
\end{align}

Define $c_{+, l,m}$ and $c_{-, l,m}$ as $\LAngularmomentum_+ \ket{l, m} = \hbar c_{+,
l,m} \ket{l, m}$ and similarly for the negative. We write $\LAngularmomentum_y =
\frac{\LAngularmomentum_+ - \LAngularmomentum_-}{2i}$ and absorb the $2i$ into
$C$ as $D = \frac{C}{2i}$ to get

\begin{align}
    H \sum_{l, m}\ket{l, m} &= 
    (A \LAngularmomentum^2 + B\LAngularmomentum_z + 
    D\LAngularmomentum_+ - D\LAngularmomentum_-) \sum_{l, m}a_{l, m}\ket{l, m}\nonumber\\
    &= \hbar \sum_{l, m}a_{l, m}((\hbar A l(l+1) + Bm)\ket{l, m} + Dc_{+, l,m}\ket{l, m+1} - Dc_{-, l,m}\ket{l, m-1}) \nonumber\\
    &= \hbar \sum_{l, m}(a_{l, m}(\hbar A l(l+1) + Bm) + Da_{l, m-1}c_{+, l,m-1} - Da_{l, m+1}c_{-, l,m+1})\ket{l, m} \\
\end{align}

This gives us a form for the matrix representation of $H$, which has diagonal
terms, and terms one above and one below the diagonal. The eigenvalues of this
matrix can be obtained (painfully) via brute force calculation.

We use $l = 1$ for this case to construct the matrix representation of $H$

\begin{align}
    H = \hbar \begin{pmatrix}
        2\hbar A - B    & -Dc_{-, 1,0}       & 0             \\   
        Dc_{+, 1,-1}    & 2\hbar A          & -Dc_{-, 1,1}   \\   
        0               & Dc_{+, 1,0}       & 2\hbar A + B    
    \end{pmatrix}~.
\end{align}

All the $c_{\pm, l,m}$ equal $\sqrt{2}$ when appearing in the matrix (zero
elements are outside $3\times 3$). And replacing back $D = \frac{C}{2i}$ we get

\begin{align}
    H = \hbar \begin{pmatrix}
        2\hbar A - B        & -\frac{C}{\sqrt{2}i}       & 0                     \\   
        \frac{C}{\sqrt{2}i} & 2\hbar A                  & -\frac{C}{\sqrt{2}i}   \\   
        0                   & \frac{C}{\sqrt{2}i}       & 2\hbar A + B    
    \end{pmatrix}~.
\end{align}

Eigenvalues of this matrix, using a symbolic solver, were found to be 

\begin{gather}
    \lambda_1 = 2 A \hbar^2~,\\
    \lambda_2 = 2 A \hbar^2 - \hbar\sqrt{B^2 + C^2}~,\\
    \lambda_3 = 2 A \hbar^2 + \hbar\sqrt{B^2 + C^2}~.
\end{gather}

At first sight, these seem very different from the previously calculated values,
but a trivial factorisation of the square root and binomial approximation to
first order gives us exactly the values calculated vio perturbation theory!