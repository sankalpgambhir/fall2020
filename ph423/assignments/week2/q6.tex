

\subsection*{Question 6}

We know that the $\mathbf{X}_i$ operators can be written in terms of the spherical tensor operators as follows: (notation is the same as that used in Shankar, Principles of Quantum Mechanics, 2ed, page 419)

\begin{equation}
    \begin{split}
        V^{+1}_1 &= \frac{i \mathbf{X}_y - \mathbf{X}_x}{\sqrt{2}} \\
        V^0_1 &= \mathbf{X}_z \\
        V^{-1}_1 &= -\frac{\mathbf{X}_x + i \mathbf{X}_y}{\sqrt{2}} 
    \end{split}
\end{equation}

Thus in general any linear combination of the $\mathbf{X}_i$s can be written in terms of the $V^i_1$s. Note that $\mathbf{\epsilon} \cdot \mathbf{X}$ is exactly such a linear combination. Thus we may write 

\begin{equation}
    \hat{O} = \mathbf{\epsilon} \cdot \mathbf{X} = \alpha_i V^i_1
\end{equation}
    
Where the $\alpha_i$ are scalars, and summation over repeated values is implied. 

Using this form we can write the transition probability for the Hydrogen atom as

\begin{equation}
    \abs{\bra{n',l',m'} \alpha_i V^i_1 \ket{n,l,m}}
\end{equation}

Now since each $V^i_1$, acting on $ \ket{n,l,m}$ can either:
\begin{itemize}
    \item Increase the value of $l$ by 1
    \item Decrease the value of $l$ by 1
    \item Keep the value of $l$ unchanged
\end{itemize}

Or give a superposition of the above. Since states of different $l$ are orthogonal, $\alpha_i V^i_1 \ket{n,l,m}$ and $\bra{n',l',m'}$ won't have any common terms unless $\abs{l - l'} = 1$ or $l = l'$. 

Thus we get the relation 

\begin{equation}
    \abs{\bra{n',l',m'} \alpha_i V^i_1 \ket{n,l,m}} = 0 
\end{equation}

Unless  $\abs{l - l'} = 1$ or $l = l'$. 

Since EM theory is invariant under parity inversion, we must require that expectation values of the dipole moment be conserved under parity inversion.

\begin{equation}
    \abs{\bra{n',l,m'} \mathbf{\epsilon} \cdot \mathbf{X} \ket{n,l,m}} = \abs{\bra{n',l',m'} P^{\dagger} \mathbf{\epsilon}\cdot \mathbf{X}  P  \ket{n,l,m}} 
\end{equation}

Since $\ket{n,l,m}$ transforms as $\ket{n,l,m} \longrightarrow (-1)^l \ket{n,l,m}$ under parity,

\begin{equation}
    (-1)^{l' + l}\bra{n',l',m'} \mathbf{\epsilon} \cdot \mathbf{X} \ket{n,l,m} = \bra{n',l',m'} P^\dagger  \mathbf{\epsilon}\cdot \mathbf{X} P \ket{n,l,m}
\end{equation}

Since $ \mathbf{X}  $ transforms as $ \mathbf{X}   \longrightarrow - \mathbf{X}  $ under parity, we get

\begin{equation}
    (-1)^{l' + l}\bra{n',l',m'} \mathbf{\epsilon} \cdot \mathbf{X} \ket{n,l,m} = -\bra{n',l',m'}  \mathbf{\epsilon}\cdot \mathbf{X}\ket{n,l,m}
\end{equation}

Hence if $l + l'$ is even (i.e. when $l = l'$), we get 

\begin{equation}
    \bra{n',l',m'} \mathbf{\epsilon} \cdot \mathbf{X} \ket{n,l,m} = 0
\end{equation}
