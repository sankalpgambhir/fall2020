

\question*{Determine the eigenvalues and eigenvectors of the 2 x 2 matrix $\paulivector$.$\normalvector$ , where $\normalvector$ is a unit vector along the $(\theta, \phi)$ direction and $\paulivector$ are the three Pauli matrices. This is basically the projection of the spin 1/2 operator (apart from $\frac{\hbar}{2}$) along the direction of the unit vector $\normalvector$. Do this in two ways: }

\parth{Doing question 2, might have issues with part (b) make sure that it's correct}

\begin{alphaparts}

\questionpart First by explicitly diagonalizing the matrix $\paulivector$.$\normalvector$.

The vector $\paulivector$ = $(\sigma_{1}, \sigma_{2}, \sigma_{3})$, where the $\sigma_{i}$ matrices are -

\begin{equation*}
    \sigma_{1} = 
    \begin{pmatrix}
        0 & 1 \\
        1 & 0
    \end{pmatrix},
    \sigma_{2} = 
    \begin{pmatrix}
        0 & -i \\
        i & 0
    \end{pmatrix},
    \sigma_{3} = 
    \begin{pmatrix}
        1 & 0 \\
        0 & -1
    \end{pmatrix}
\end{equation*}

Now we need to figure out what $\normalvector$ is. The unit vector points along the $(\theta, \phi)$ direction. This is nothing but the unit vector $\hat{\mathbf{r}}$ in Polar co-ordinates.
\begin{equation*}
    \normalvector = \boldsymbol{\hat{r}} = \textrm{cos}(\phi)\textrm{sin}(\theta)\mathbf{\hat{i}} + \textrm{sin}(\phi)\textrm{sin}(\theta)\mathbf{\hat{j}} + \textrm{cos}(\theta)\mathbf{\hat{k}}
\end{equation*}

Thus, $\hat{\mathbf{n}}$ = $(\textrm{cos}(\phi)\textrm{sin}(\theta), \textrm{sin}(\phi)\textrm{sin}(\theta), \textrm{cos}(\theta))$. We know that $\mathbf{a}.\mathbf{b}$ = $a_{i}b_{i}$ (implicit summation over i)\\
Thus, $\paulivector$.$\normalvector$ = $\sigma_{i}n_{i}$.
\begin{gather*}
    \paulivector.\normalvector = \textrm{cos}(\phi)\textrm{sin}(\theta)
    \begin{pmatrix}
        0 & 1 \\
        1 & 0
    \end{pmatrix} + \textrm{sin}(\phi)\textrm{sin}(\theta)
    \begin{pmatrix}
        0 & -i \\
        i & 0
    \end{pmatrix} + \textrm{cos}(\theta)
    \begin{pmatrix}
        1 & 0 \\
        0 & -1
    \end{pmatrix} \\
    \therefore \paulivector.\normalvector = \textrm{sin}(\theta)
    \begin{pmatrix}
        0 & \textrm{cos}(\phi) - i*\textrm{sin}(\phi) \\
        \textrm{cos}(\phi) + i*\textrm{sin}(\phi) & 0
    \end{pmatrix} + \textrm{cos}(\theta)
    \begin{pmatrix}
        1 & 0 \\
        0 & -1
    \end{pmatrix} \\
    = \textrm{sin}(\theta)
    \begin{pmatrix}
        0 & e^{-i\phi} \\
        e^{i\phi} & 0
    \end{pmatrix} + \textrm{cos}(\theta)
    \begin{pmatrix}
        1 & 0 \\
        0 & -1
    \end{pmatrix} = 
    \begin{pmatrix}
        \textrm{cos}(\theta) & \textrm{sin}(\theta)e^{-i\phi} \\
        \textrm{sin}(\theta)e^{i\phi} & -\textrm{cos}(\theta)
    \end{pmatrix}
\end{gather*}

To find the eigenvalues and eigenvectors, we now need to diagonalize this matrix. Let the eigenvalues be represented by $\lambda$. The characteristic polynomial takes the following form.
\begin{gather*}
    (\textrm{cos}(\theta) - \lambda)(-\textrm{cos}(\theta) - \lambda) - \textrm{sin}(\theta)e^{-i\phi}*\textrm{sin}(\theta)e^{i\phi} = 0 \\
    \therefore -\textrm{cos}^{2}(\theta) + \lambda^{2} - \textrm{sin}^{2}(\theta) = 0  \Rightarrow \lambda^{2} - 1 = 0 \\
    \therefore \lambda = \pm 1
\end{gather*}

for $\lambda = 1$, let the eigenvector be $\mathbf{v_{1}} = (v_{1,1}, v_{1,2})$, thus 

\begin{gather*}
    \begin{pmatrix}
        \textrm{cos}(\theta) & \textrm{sin}(\theta)e^{-i\phi} \\
        \textrm{sin}(\theta)e^{i\phi} & -\textrm{cos}(\theta)
    \end{pmatrix}
    \begin{pmatrix}
        v_{1,1} \\
        v_{1,2}
    \end{pmatrix} = 
    \begin{pmatrix}
        v_{1,1} \\
        v_{1,2}
    \end{pmatrix} \\
    \therefore \textrm{cos}(\theta)*v_{1,1} + \textrm{sin}(\theta)e^{-i\phi}*v_{1,2} = v_{1,1} \; , \; \textrm{sin}(\theta)e^{i\phi}*v_{1,1} - \textrm{cos}(\theta)*v_{1,2} = v_{1,2} \\
    v_{1,2} = e^{i\phi}\frac{\textrm{sin}(\theta)}{(\textrm{cos}(\theta) + 1)}*v_{1,1} \\
\end{gather*}

Thus, for eigenvalue $\lambda = 1$, the eigenvector $\mathbf{v_{1}} = (v_{1,1}, e^{i\phi}\frac{\textrm{sin}(\theta)}{(\textrm{cos}(\theta) + 1)}*v_{1,1})$

Likewise, for $\lambda = -1$, let the eigenvector be $\mathbf{v_{2}} = (v_{2,1}, v_{2,2})$, thus

\begin{gather*}
    \begin{pmatrix}
        \textrm{cos}(\theta) & \textrm{sin}(\theta)e^{-i\phi} \\
        \textrm{sin}(\theta)e^{i\phi} & -\textrm{cos}(\theta)
    \end{pmatrix}
    \begin{pmatrix}
        v_{2,1} \\
        v_{2,2}
    \end{pmatrix} = 
    \begin{pmatrix}
        -v_{2,1} \\
        -v_{2,2}
    \end{pmatrix} \\
    \therefore \textrm{cos}(\theta)*v_{2,1} + \textrm{sin}(\theta)e^{-i\phi}*v_{2,2} = -v_{2,1}, \textrm{sin}(\theta)e^{i\phi}*v_{2,1} - \textrm{cos}(\theta)*v_{2,2} = -v_{2,2} \\
    v_{2,2} = e^{i\phi}\frac{\textrm{sin}(\theta)}{(1 - \textrm{cos}(\theta))}*v_{2,1} \\
\end{gather*}

Thus, for eigenvalue $\lambda = -1$, the eigenvector $\mathbf{v_{2}} = (v_{2,1}, e^{i\phi}\frac{\textrm{sin}(\theta)}{(1 - \textrm{cos}(\theta))}*v_{2,1})$.\\
We thus have our two eigenvalues ($\pm 1$) and our two eigenvectors ($\mathbf{v_{1}}$ and $\mathbf{v_{2}}$) \\

\questionpart By rotating the spinor pointing initially along the $+\hat{\mathbf{z}}$ axis direction by appropriate angles, u\textrm{sin}g the appropriate rotation operator. Convince yourself that one has to rotate by an angle $\theta$ counterclockwise around the \textit{y}-axis and then by $\phi$ around the \textit{z}-axis. Apart from overall phases, is the resultant spinor the same as the spin up eigenvector obtained in part \textbf{(a)}? 

Let's start with the spinor pointing in the +\textit{z}-direction.

\begin{equation*}
    \ket{s_{z} = +\frac{\hbar}{2}} = 
    \begin{bmatrix}
        1 \\
        0
    \end{bmatrix}, \; s.t \;  S_{z}\ket{s_{z} = +\frac{\hbar}{2}} = +\frac{\hbar}{2}\ket{s_{z} = +\frac{\hbar}{2}}
\end{equation*}

If we apply consecutive rotation operators, we should be able to rotate this spinor into a general state, pointing in an arbitrary direction $\normalvector$, where $\normalvector$ points in the $(\theta, \phi)$ direction. \\
We first rotate this spinor by $\theta$ around the \textit{y}-axis, and then by $\phi$ around the \textit{z}-axis. The axis of spin now points in the direction $\normalvector$. Thus - 

\begin{equation*}
    \ket{\hat{n}+} = U[R(\phi\hat{\mathbf{z}})]U[R(\theta\hat{\mathbf{y}})]
    \begin{bmatrix}
        1 \\
        0
    \end{bmatrix}
\end{equation*}

To find the explicit form of $\ket{\hat{n}+}$, we'll need the forms of the unitary matrices $U[R(\phi\hat{\mathbf{z}})]$ and $U[R(\theta\hat{\mathbf{y}})]$. We'll use the result given in Shankar - 

\begin{equation*}
        U[R(\boldsymbol{\theta}] = \textrm{cos}\frac{\theta}{2}I - i\textrm{sin}\frac{\theta}{2}(\hat{\theta}.\boldsymbol{\sigma})
\end{equation*}

Looking at the particular case of rotation around \textit{y}-axis by amount $\theta$ and then subsequently around \textit{z}-axis by amount $\phi$ - 

\begin{equation*}
    \begin{split}
        U[R(\theta\hat{\mathbf{y}})]
        \begin{bmatrix}
            1 \\
            0
        \end{bmatrix} & = \left[ \textrm{cos}\frac{\theta}{2}I - i \, \textrm{sin}\frac{\theta}{2}\sigma_{y}\right]
        \begin{bmatrix}
            1 \\
            0
        \end{bmatrix} \\
        & = 
        \begin{bmatrix}
            \textrm{cos}\frac{\theta}{2} \\
            0
        \end{bmatrix} - i \, \textrm{sin}\frac{\theta}{2}
        \begin{bmatrix}
            0 & -i \\
            i & 0
        \end{bmatrix}
        \begin{bmatrix}
            1 \\
            0
        \end{bmatrix} \\
        & = 
        \begin{bmatrix}
            \textrm{cos}\frac{\theta}{2} \\
            \textrm{sin}\frac{\theta}{2}
        \end{bmatrix}
    \end{split}
\end{equation*}

Applying rotation around \textit{z}-axis by amount $\phi$ now, we get 
\begin{equation*}
    \begin{split}
        U[R(\phi\hat{\mathbf{z}})]
        \begin{bmatrix}
            \textrm{cos}\frac{\theta}{2} \\
            \textrm{sin}\frac{\theta}{2}
        \end{bmatrix} & = \left[ \textrm{cos}\frac{\phi}{2}I - i \, \textrm{sin}\frac{\phi}{2}\sigma_{z} \right]
        \begin{bmatrix}
            \textrm{cos}\frac{\theta}{2} \\
            \textrm{sin}\frac{\theta}{2}
        \end{bmatrix} \\
        & = 
        \begin{bmatrix}
            \textrm{cos}\frac{\phi}{2}\textrm{cos}\frac{\theta}{2} \\
            \textrm{cos}\frac{\phi}{2}\textrm{sin}\frac{\theta}{2}
        \end{bmatrix} - i \, \textrm{sin}\frac{\phi}{2}
        \begin{bmatrix}
            1 & 0 \\
            0 & -1
        \end{bmatrix}
        \begin{bmatrix}
            \textrm{cos}\frac{\theta}{2} \\
            \textrm{sin}\frac{\theta}{2}
        \end{bmatrix} \\
        & = 
        \begin{bmatrix}
            \textrm{cos}\frac{\theta}{2} \left( \textrm{cos}\frac{\phi}{2} - i \, \textrm{sin}\frac{\phi}{2}\right) \\
            \textrm{sin}\frac{\theta}{2} \left( \textrm{cos}\frac{\phi}{2} + i \, \textrm{sin}\frac{\phi}{2} \right)
        \end{bmatrix} \\
        & = 
        \begin{bmatrix}
            \textrm{cos}\frac{\theta}{2}e^{-i\frac{\phi}{2}} \\
            \textrm{sin}\frac{\theta}{2}e^{i\frac{\phi}{2}}
        \end{bmatrix}
    \end{split}
\end{equation*}

This gives us a spinor $s_{n} = (s_{n1}, s_{n2}) = (\textrm{cos}\frac{\theta}{2}e^{-i\frac{\phi}{2}}, \textrm{sin}\frac{\theta}{2}e^{i\frac{\phi}{2}})$. If we recall our $\mathbf{v_{1}} = (v_{1,1}, v_{1,2})$ from part \textbf{(a)}, we recall the relation we obtained at the end.

\begin{equation*}
    v_{1,2} = e^{i\phi}\frac{\textrm{sin}(\theta)}{(\textrm{cos}(\theta) + 1)}*v_{1,1}
\end{equation*}

Substituting $v_{1,1} = s_{n1} = \textrm{cos}\frac{\theta}{2}e^{-i\frac{\phi}{2}}$ (as our final spinor seems to suggest), we get - 

\begin{equation*}
    \begin{split}
        v_{1,2} & = e^{i\phi}\frac{\textrm{sin}(\theta)}{(\textrm{cos}(\theta) + 1)}*v_{1,1} \\
        & = e^{i\phi}\frac{\textrm{sin}(\theta)}{(\textrm{cos}(\theta) + 1)}*\textrm{cos}\frac{\theta}{2}e^{-i\frac{\phi}{2}} \\
    \end{split}
\end{equation*}

Recall $1 + \textrm{cos}(A) = 2*\textrm{cos}^{2}(\frac{A}{2})$ and $\textrm{sin}(A) = 2*\textrm{sin}(\frac{A}{2})\textrm{cos}(\frac{A}{2})$

\begin{equation*}
    \begin{split}
        e^{i\phi}\frac{\textrm{sin}(\theta)}{(\textrm{cos}(\theta) + 1)}*\textrm{cos}\frac{\theta}{2}e^{-i\frac{\phi}{2}} & = e^{i\frac{\phi}{2}}\frac{\textrm{sin}(\theta)}{2cos^{2}(\frac{\theta}{2})}*\textrm{cos}\frac{\theta}{2} \\
        & = e^{i\frac{\phi}{2}}\frac{2\textrm{sin}(\frac{\theta}{2})\textrm{cos}(\frac{\theta}{2})}{2cos^{2}(\frac{\theta}{2})}*\textrm{cos}\frac{\theta}{2} \\
        & = e^{i\frac{\phi}{2}}\textrm{sin}(\frac{\theta}{2}) = s_{n2}
    \end{split}
\end{equation*}

Therefore, apart from phase factors, the resultant spinor is the same as the spin up eigenvector we got in part \textbf{(a)}.


\end{alphaparts}