\newcounter{draft}
\setcounter{draft}{1}

% define general format and includes for all assignments
\documentclass[10pt, largemargins]{../../../common/homework}
\usepackage{geometry}

% import req basic packages
\usepackage[dvipsnames]{xcolor}
\usepackage[tracking=true]{microtype}
\usepackage{amsmath, amssymb}
\usepackage{subcaption}
\usepackage{graphicx}
\usepackage{ifthen}
\usepackage{lineno}
\usepackage[utf8]{inputenc}
\usepackage{physics}
\usepackage{karnaugh-map}
\usepackage{circuitikz}

% tikz stuffs
\usepackage{tikz}
\usepackage{tkz-orm}
\usetikzlibrary{arrows,shapes,automata,backgrounds,petri,decorations.markings}
\usetikzlibrary{matrix,positioning,decorations.pathreplacing,calc,tikzmark}

% customised links
\usepackage{hyperref}
\hypersetup{
    colorlinks,
    linkcolor={red!50!black},
    citecolor={blue!50!black},
    urlcolor={blue!80!black}
}

% setup fancy author list
\usepackage{array}
\def\and{                  % \begin{tabular}
  \end{tabular}
  \hskip 2em 
  \begin{tabular}[t]{>{\centering\arraybackslash}p{0.2\textwidth}}}%   % \end{tabular}

% set font
\usepackage[T1]{fontenc}
\usepackage{baskervillef}
\usepackage[varqu,varl,var0]{inconsolata}
\usepackage[scale=.95,type1]{cabin}
\usepackage[baskerville,vvarbb]{newtxmath}
\usepackage[cal=boondoxo]{mathalfa}


% reviews and notes
\ifthenelse{\value{draft} > 0}
{
  \linenumbers
  \newcommand{\parth}[1]{\textcolor{OliveGreen}{$\left[\textnormal{Parth: #1}\right]$}}
  \newcommand{\sahas}[1]{\textcolor{TealBlue}{$\left[\textnormal{Sahas: #1}\right]$}}
  \newcommand{\sankalp}[1]{\textcolor{WildStrawberry}{$\left[\textnormal{Sankalp: #1}\right]$}}
}
{
  \newcommand{\parth}[1]{}
  \newcommand{\sahas}[1]{}
  \newcommand{\sankalp}[1]{}
}

% homework class and document settings
\newcommand{\hwauthorlist}{ Sankalp Gambhir \\ \texttt{180260032}
} % author list for first page only, slightly jank
\newcommand{\hwname}{Sankalp Gambhir}
\newcommand{\hwemail}{\texttt{180260032}}
\newcommand{\hwtype}{Assignment}
\newcommand{\hwnum}{0} % default number
\newcommand{\hwdate}{\today} % default to today if no submission date specified 
\newcommand{\hwclass}{NULL}
\newcommand{\hwlecture}{} % unused, but needed
\newcommand{\hwsection}{} % unused, but needed


\begin{document}

    \parth{Doing question 5, I'm not spending as much time on this as question 2}

    \question*{Prove that any function of the radial coordinate $f(r)$ where $r = |\mathbf{r}|$ and \textbf{X} · \textbf{P}, where \textbf{X} and \textbf{P} are the position and momentum operators, are both scalar operators.}

    Under a symmetry operator $U$, operators change as $\mathcal{O}' = U^{\dagger}\mathcal{O}U$. A scalar operator being one which is invariant under rotations, i.e
    \begin{equation*}
        S' = U^{\dagger}[R]SU[R] = S
    \end{equation*}

    where $U(R(\boldsymbol{\alpha}) = e^{-\frac{i}{\hbar}\boldsymbol{\alpha}.\mathbf{J}})$. \\
    By considering infinitesimal rotations $\boldsymbol{\alpha} = \boldsymbol{\epsilon}$, we have 

    \begin{equation*}
        U[R(\boldsymbol{\alpha})] = \left( 1 - \frac{i}{\hbar}\epsilon_{i}J_{i}\right)
    \end{equation*}

    Thus, our definition for a scalar operator becomes - 

    \begin{equation*}
        S' = \left( 1 + \frac{i}{\hbar}\epsilon_{i}J_{i}\right)S\left( 1 - \frac{i}{\hbar}\epsilon_{i}J_{i}\right) = S
    \end{equation*}

    which gives us $\frac{i}{\hbar}\epsilon_{i}\left[ J_{i},S \right] = 0$. Since $\boldsymbol{\epsilon}$ was an arbitrary choice, we have 

    \begin{equation*}
        \left[ J_{i},S \right] = 0
    \end{equation*}

    as our definition of a scalar operator.

    Considering $f(r)$, where $r = |\mathbf{r}|$ as our operator.

    \begin{equation*}
        \left[ J_{i},f(r) \right] = \left[ J_{i},r \right]*f'(r)
    \end{equation*}

    $r = \sqrt{\sum_{i=1}^{3}X_{i}^{2}}$, Thus

    \begin{equation*}
        \left[ J_{i},r \right] = \left[ J_{i},X_{1} \right]*\frac{X_{1}}{r} + \left[ J_{i},X_{2} \right]*\frac{X_{2}}{r} + \left[ J_{i},X_{3} \right]*\frac{X_{3}}{r}
    \end{equation*}

    we know that $\left[ J_{i}, X_{j}\right] = i\hbar\epsilon_{ijl}X_{l}$. Thus

    \begin{equation*}
        \left[ J_{i},r \right] = \left[ J_{i},X_{j} \right]*\frac{X_{j}}{r} = \frac{1}{r}\left(i\hbar\epsilon_{ijl}X_{l}X_{j}\right)
    \end{equation*}

    \begin{equation*}
        \epsilon_{ijl}X_{l}X_{j} = \left[ X_{l},X_{j}\right] = 0 (l \neq j) \Rightarrow \left[ J_{i},r \right] = 0
    \end{equation*}

    Thus, since $\left[ J_{i},r \right] = 0$, we have $\left[ J_{i},f(r) \right] = \left[ J_{i},r \right]*f'(r) = 0*f'(r) = 0$.\\
    Thus, $f(r)$ is a scalar operator.\\

    Now considering $O =$ \textbf{X} . \textbf{P} as our operator, we need to show $ \left[ J_{i},O \right] = 0 $

    \begin{equation*}
        \mathbf{X}.\mathbf{P} = X_{i}P_{i} \qquad {\text{implicit summation}}
    \end{equation*}
        \begin{equation*}
        \begin{split}
            \therefore \left[ J_{i},O \right] & = \left[ J_{i}, X_{j}P_{j} \right] \\
            & = \left[ J_{i}, X_{j} \right]P_{j} + X_{j}\left[ J_{i}, P_{j} \right] \\
            & = i\hbar\epsilon_{ijl}(X_{l}P_{j} + X_{j}P_{l})
        \end{split}
    \end{equation*}

    Now, $\epsilon_{ijl}X_{l}P_{j} = [X_{l}, P_{j}]$ for $l \neq j$, but $[X_{l}, P_{j}] = 0, l \neq j$. Thus

    \begin{equation*}
            i\hbar\epsilon_{ijl}(X_{l}P_{j} + X_{j}P_{l}) = 0 \Rightarrow [J_{i}, O] = 0
    \end{equation*}

    Since $[J_{i}, O] = 0$, we can say that the operator $O$ is a scalar operator.\\
    Thus, \textbf{X}.\textbf{P} is a scalar operator

\end{document}