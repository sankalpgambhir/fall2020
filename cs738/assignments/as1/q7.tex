\question

The file was downloaded from Piazza. The mutex condition was verified to hold as
written in the example. The nospurious condition was coded as follows and found
to hold as well.

\begin{lstlisting}
    INVARSPEC
    (
    (e1.ack-out -> e1.Request)
    &(e2.ack-out -> e2.Request)
    &(e3.ack-out -> e3.Request)
    &(e4.ack-out -> e4.Request)
    &(e5.ack-out -> e5.Request)
    )
\end{lstlisting}

The property such that if a request is held on, it is eventually acknowledged, is given by

\begin{equation}
    \globally ((\globally \texttt{req}_i) \Rightarrow \finally\texttt{ack}_i) 
\end{equation}

It was coded in as

\begin{lstlisting}   
LTLSPEC
    G(
    ((G e1.Request) -> (F e1.ack-out))
    &((G e3.Request) -> (F e3.ack-out))
    )
\end{lstlisting}

However, NuSMV verified this specification as being true for the model.

The no loss invariant was coded in 

\begin{lstlisting}
INVARSPEC
    (
       (!e1.loss)
      &(!e2.loss)
      &(!e3.loss)
      &(!e4.loss)
      &(!e5.loss)
    )
\end{lstlisting}

and NuSMV found it to be false.

The 3 consecutive loss property is stated as 

\begin{equation}
    \always \globally \lnot (\texttt{loss} \land \always\lnext(\texttt{loss} \land \always\lnext\texttt{loss}))
\end{equation}

and was coded as 

\begin{lstlisting}
CTLSPEC
    (
       (AG (! (e1.loss & AX(e1.loss & AX(e1.loss)))))
      &(AG (! (e2.loss & AX(e2.loss & AX(e2.loss)))))
      &(AG (! (e3.loss & AX(e3.loss & AX(e3.loss)))))
      &(AG (! (e4.loss & AX(e4.loss & AX(e4.loss)))))
      &(AG (! (e5.loss & AX(e5.loss & AX(e5.loss)))))
    )
\end{lstlisting}

with NuSMV once again verifying it to be true.

Finally modifying the circuit to use the value of incoming token instead of last
token value

\begin{lstlisting}
    next(Persistent) := Request & (Persistent | Token);

    -- CHANGED TO -->

    next(Persistent) := Request & (Persistent | token-in);
\end{lstlisting}

However, curiously, running this, NuSMV reported \textbf{no change in specification
validity.}