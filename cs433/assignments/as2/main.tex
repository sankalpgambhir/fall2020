\newcounter{draft}
\setcounter{draft}{0}

% define general format and includes for all assignments
\documentclass[10pt, largemargins]{../../../common/homework}
\usepackage{geometry}

% import req basic packages
\usepackage[dvipsnames]{xcolor}
\usepackage[tracking=true]{microtype}
\usepackage{amsmath, amssymb}
\usepackage{subcaption}
\usepackage{graphicx}
\usepackage{ifthen}
\usepackage{lineno}
\usepackage[utf8]{inputenc}
\usepackage{physics}
\usepackage{karnaugh-map}
\usepackage{circuitikz}

% tikz stuffs
\usepackage{tikz}
\usepackage{tkz-orm}
\usetikzlibrary{arrows,shapes,automata,backgrounds,petri,decorations.markings}
\usetikzlibrary{matrix,positioning,decorations.pathreplacing,calc,tikzmark}

% customised links
\usepackage{hyperref}
\hypersetup{
    colorlinks,
    linkcolor={red!50!black},
    citecolor={blue!50!black},
    urlcolor={blue!80!black}
}

% setup fancy author list
\usepackage{array}
\def\and{                  % \begin{tabular}
  \end{tabular}
  \hskip 2em 
  \begin{tabular}[t]{>{\centering\arraybackslash}p{0.2\textwidth}}}%   % \end{tabular}

% set font
\usepackage[T1]{fontenc}
\usepackage{baskervillef}
\usepackage[varqu,varl,var0]{inconsolata}
\usepackage[scale=.95,type1]{cabin}
\usepackage[baskerville,vvarbb]{newtxmath}
\usepackage[cal=boondoxo]{mathalfa}


% reviews and notes
\ifthenelse{\value{draft} > 0}
{
  \linenumbers
  \newcommand{\parth}[1]{\textcolor{OliveGreen}{$\left[\textnormal{Parth: #1}\right]$}}
  \newcommand{\sahas}[1]{\textcolor{TealBlue}{$\left[\textnormal{Sahas: #1}\right]$}}
  \newcommand{\sankalp}[1]{\textcolor{WildStrawberry}{$\left[\textnormal{Sankalp: #1}\right]$}}
}
{
  \newcommand{\parth}[1]{}
  \newcommand{\sahas}[1]{}
  \newcommand{\sankalp}[1]{}
}

% homework class and document settings
\newcommand{\hwauthorlist}{ Sankalp Gambhir \\ \texttt{180260032}
} % author list for first page only, slightly jank
\newcommand{\hwname}{Sankalp Gambhir}
\newcommand{\hwemail}{\texttt{180260032}}
\newcommand{\hwtype}{Assignment}
\newcommand{\hwnum}{0} % default number
\newcommand{\hwdate}{\today} % default to today if no submission date specified 
\newcommand{\hwclass}{NULL}
\newcommand{\hwlecture}{} % unused, but needed
\newcommand{\hwsection}{} % unused, but needed

\usepackage{csvsimple}

% homework class and document settings
\renewcommand{\hwauthorlist}{   Parth Sastry    \\ \texttt{180260026} \and
                                Sankalp Gambhir \\ \texttt{180260032}
} % author list for first page only, slightly jank
\renewcommand{\hwname}{Parth Sastry, Sankalp Gambhir}
\renewcommand{\hwemail}{\texttt{180260026, 32}}
\renewcommand{\hwclass}{CS433}
\renewcommand{\hwtype}{Assignment}
\renewcommand{\hwnum}{2}

\begin{document}
    \maketitle

    \begin{figure}[!hp]

        \centering
        \scalebox{1.5}{
        
    \begin{tikzpicture}
        \pgfplotsset{
            scale only axis,
            compat=1.3,
        }
        
        \begin{axis}[
            xlabel = Time,
            ylabel = Problems,
            xmode = log,
            ymode = log,
            typeset ticklabels with strut,
            legend style={at={(0.5,-0.3)},anchor=north}
        ]
    
        \addplot table [col sep=comma] {data/out.txt.csv};
        \addplot table [col sep=comma] {data/out600.txt.csv};
        \addplot table [col sep=comma] {data/out700.txt.csv};
        \addplot table [col sep=comma] {data/out800.txt.csv};
        \addplot table [col sep=comma] {data/out900.txt.csv};
        \addplot table [col sep=comma] {data/out1000.txt.csv};
        \addplot table [col sep=comma] {data/out1000n.txt.csv};
        \addplot table [col sep=comma] {data/out500n.txt.csv};
        \label{axis:prediction}
        
        % plot 1 legend entry
        \addlegendimage{/pgfplots/refstyle=axis:prediction}
        \addlegendentry{Scorefactor 500}   
        \addlegendentry{Scorefactor 600}   
        \addlegendentry{Scorefactor 700}   
        \addlegendentry{Scorefactor 800}   
        \addlegendentry{Scorefactor 900}   
        \addlegendentry{Scorefactor 1000}   
        \addlegendentry{Scorefactor 1000, Score false}   
        \addlegendentry{Scorefactor 500, Score false}   
         
        \end{axis}
    
    \end{tikzpicture}}
    \caption{Time taken versus problems solved}
    \label{fig:time}
\end{figure}

    As we see in \autoref{fig:time}, the runs with \texttt{score = false} come out on top at the start, but begin
    to dwindle with more complex problems. This suggests that the score calculations add 
    unnecessary complexity to smaller problems and removing it adds to the performance, while
    in large problems with more conflicts it would help settle in on a result quicker.

    We are not particularly sure about the \texttt{scorefactor} parameter based results. We got 
    different results for the times while changing \texttt{scorefactor} even though we had
    asked to not use EVSIDS scores, consistently across tests we performed. The lower \texttt{scorefactor}
    performed better here. Based on these and the paper on techniques used here by \href{http://fmv.jku.at/papers/BiereFroehlich-SAT15.pdf}{Biere et al},
    creators of CaDiCaL that this infact corresponds to the damping factor \emph{f} used in the paper.
    
    Across several runs to verify our results, we did not observe much change in time compared to these
    results. Here, we see that the reds, 600 and 1000 both perform about the same, with the blues 500 and 900 being a 
    bit better, and black/browns 700 and 800 performing better than the rest. While our data is limited by
    the scope of available computing power, this does suggest that values of \texttt{scorefactor} on either 
    ends of the spectrum (allowed values 500 to 1000 per \href{https://www.mankier.com/1/cadical}{documentation})
    seem to degrade performance, perhaps by causing too much decay and flattening the scores, or causing too little,
    and letting scores build up.

\end{document}