\begin{alphaparts}

    \questionpart 
    Consider the character table of $\cyclic_{3v}$

    \begin{center}
        \begin{tabular}[ht]{c | c c c}
            $\cyclic_{3v}$  &   e   &   $\cyclic_{3}$   &   $\reflection_v$ \\ \hline
            $A_1$           &   1   &   1               &   1               \\
            $A_2$           &   1   &   1               &   -1              \\
            $E$             &   2   &   -1              &   0               
        \end{tabular}
    \end{center}

    Using the Great Orthogonality Theorem,

    \begin{equation}
        a_\alpha = \frac{1}{\abs*{G}} \cdot \left( \sum_i n_i \chi(i) \chi_\alpha(i)\right)~,
    \end{equation}

    and the characters given, we get,

    \begin{align}
        a_{A_1} &= \frac{1}{6} \cdot \left( 1 \cdot 1 \cdot 7 + 2 \cdot 1 \cdot 1 + 3 \cdot 1 \cdot (-3)  \right) = 0~,\\
        a_{A_2} &= \frac{1}{6} \cdot \left( 1 \cdot 1 \cdot 7 + 2 \cdot 1 \cdot 1 + 3 \cdot (-1) \cdot (-3)  \right) = 3~, \textnormal{and}\\
        a_{E}   &= \frac{1}{6} \cdot \left( 1 \cdot 2 \cdot 7 + 2 \cdot (-1) \cdot 1 + 3 \cdot 0 \cdot (-3)  \right) = 2~.
    \end{align}

    \questionpart 
    $\cyclic_s$ is abelian, so it has no symmetric degeneracies. All 2-dimensional degeneracies
    of $\cyclic_{3v}$ split into two non-degenerate states.

    \setcounter{partCounter}{6} % skip to g, h

    \questionpart 
    $\symmetric(4)$ has irreps with dimensions \{1, 1, 2, 3, 3\}. They are represented by the Young's diagrams given 
    below, left to right:

    % draw young diagrams

    \begin{figure}[ht]
        \centering
        \begin{subfigure}[b]{0.15\textwidth}
            \ydiagram{4}
        \end{subfigure}
        \begin{subfigure}[b]{0.15\textwidth}
            \ydiagram{1, 1, 1, 1}
        \end{subfigure}
        \begin{subfigure}[b]{0.15\textwidth}
            \ydiagram{2, 2}
        \end{subfigure}
        \begin{subfigure}[b]{0.15\textwidth}
            \ydiagram{3, 1}
        \end{subfigure}
        \begin{subfigure}[b]{0.15\textwidth}
            \ydiagram{2, 1, 1}
        \end{subfigure}
    \end{figure}

    \questionpart 
    The electric dipole moment and magnetic dipole moment operators are given by:
    
    \begin{equation}
        \hat \mu_E = (q\hat X, q\hat Y, q\hat Z), \hat \mu_B = \frac{g}{2m} (\hat L_x, \hat L_y, \hat L_z)
    \end{equation}

    with 
    \begin{align}
        \hat Z \in A_1 \\
        \hat X, \hat Y, \hat L_x, \hat L_y \in A_2 \\
        \hat L_z \in E
    \end{align}

    following the product rules

    \begin{align}
        A_1 \otimes \Gamma = \Gamma \textnormal{ ~for any irrep } \Gamma \\
        A_2 \otimes A_2 = A_1 \\
        A_2 \otimes E = E \\
        E \otimes E = A_1 \oplus A_2 \oplus E~.
    \end{align}

    So in any of the required integral computations, we will have a term from $A_1$ and thus,
    the questioned transitions will be disallowed.

\end{alphaparts}