\begin{alphaparts}

    \questionpart \textbf{3.}\\
    This is the group $SO(2, 1)$ of Lorentz transformations. It has 3
    generators, given by boosts along either of the axes, and rotation in the
    plane.

    \questionpart \textbf{$\sqrt{\frac{-2\hbar E}{\mu}}$~.}\\
    We use the fact that the Lie bracket is bilinear, and substitute the given
    value of the scalar-free Lie Bbracket to obtain the desired expression for a.

    \questionpart \textbf{$L_y$.}\\
    The generators mentioned are given by
    \[L_z = \begin{pmatrix}
        0 & 0 & 0 & 0 \\
        0 & 0 &-1 & 0 \\
        0 & 1 & 0 & 0 \\
        0 & 0 & 0 & 0
    \end{pmatrix}, \textnormal{and }
    K_x = \begin{pmatrix}
        0 & 1 & 0 & 0 \\
        1 & 0 & 0 & 0 \\
        0 & 0 & 0 & 0 \\
        0 & 0 & 0 & 0
    \end{pmatrix} \]

    and their commutator is easily calculated to be $K_y$, where in the same
    basis
    \[K_y = \begin{pmatrix}
        0 & 0 & 1 & 0 \\
        0 & 0 & 0 & 0 \\
        1 & 0 & 0 & 0 \\
        0 & 0 & 0 & 0
    \end{pmatrix}~.\]

    \questionpart \textbf{${12, 0, 2}$.}\\
    The characters are easily obtained by direct application of the required
    elements. The identity element preserves all coordinates, 12. The rotation
    does not preserve the position onf any atom, while the reflection preserves
    two.

    \questionpart \textbf{22.}\\
    The number of independent parameters in an $n \times n$ matrix is $2n^2$.
    The unitarity condition imposes a total of $\frac{n(n+1)}{2}$ constraints on
    this system ($n^2$ equations, and a strictly triangular part of size
    $\frac{n(n-1)}{2}$ removed, being identical due to adjoint invariance of the
    unitarity condition) remove the same number of independent parameters.
    Leaving us with $2n^2 - \frac{n(n+1)}{2}$ parameters, being 22 when
    evaluated for $n = 4$.

\end{alphaparts}