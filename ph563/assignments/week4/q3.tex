\begin{alphaparts}

    \questionpart 
    The symmetry group for the given figure is clearly $\cyclic_{4v}$. The
    character table for this group is

    \begin{center}
        \begin{tabular}{c|c c c c c c}
            $C_{4v}$ & $E$ & $2C_4$(y) & $C_2$ & $2\sigma_v$ & $2\sigma_d$ & basis \\
            \hline
            $A_1$ & 1 & 1 & 1 & 1 & 1 & $y$ \\
            $A_2$ & 1 & 1 & 1 & -1 & -1 & $R_y$ \\
            $B_1$ & 1 & -1 & 1 & 1 & -1 & \\
            $B_2$ & 1 & -1 & 1 & -1 & 1 & \\
            $E$ & 2 & 0 & -2 & 0 & 0 & ($x, z$), ($R_x, R_z$)\\
        \end{tabular}
    \end{center}

    The characters for the classes in the columns, in order, would be, 8, as the
    identity preserves everything, 0, 0, and 0, since none of the reflections
    and rotations preserve the atoms' positions.

    Using the character orthogonality relations, we get 

    \begin{equation}
        \Gamma = A_1 \oplus A_2 \oplus B_1 \oplus B_2 \oplus 2E~.
    \end{equation}


    \questionpart 
    Using the same process as in part a, we can calculate the decomposition, this
    time of course, using the character table of $T_d$ instead.

    \[\chi^{\Gamma^\alpha} = \{9, 0, 1, -1, 3\}\]

    for the conjugacy classes $E, \cyclic_3, \cyclic_2, \rotoreflection_4, \reflection_d$.

    And we get $\Gamma = A_1 \oplus A_2 \oplus B_1 \oplus B_2 \oplus 2E~.$

    \questionpart \textbf{(i)}
    The generators mentioned are given by
    \[L_z = \begin{pmatrix}
        0 & 0 & 0 & 0 \\
        0 & 0 &-1 & 0 \\
        0 & 1 & 0 & 0 \\
        0 & 0 & 0 & 0
    \end{pmatrix}, \textnormal{and }
    K_x = \begin{pmatrix}
        0 & 1 & 0 & 0 \\
        1 & 0 & 0 & 0 \\
        0 & 0 & 0 & 0 \\
        0 & 0 & 0 & 0
    \end{pmatrix} \]

    and their commutator is easily calculated to be $K_y$, where in the same
    basis
    \[K_y = \begin{pmatrix}
        0 & 0 & 1 & 0 \\
        0 & 0 & 0 & 0 \\
        1 & 0 & 0 & 0 \\
        0 & 0 & 0 & 0
    \end{pmatrix}~.\]

\end{alphaparts}