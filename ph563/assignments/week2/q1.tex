\begin{alphaparts}
    \questionpart
    \textbf{ACD.} The group is defined by the direct product in A.
    Since $\dihedral_n$ is not Abelian for $n \geq 3$, $\dihedral_3$ isn't
    and neither is its direct product, hence C. Finally, since $\dihedral_3 \subset \dihedral_6$
    their direct products with $\reflection_h$ will follow their hierarchy, so we get D.

    \questionpart
    \textbf{B.} The identity element is not conjugate to anything, obviously. The next conjugacy
    class is given by rotational symmteries while preserving the order of atoms. Finally, we get 
    the last set, with rotational symmetries of the molecule with an altered atomic order.

    \questionpart
    \textbf{AB.} The symmetric irreducible representation invariably has dimension 1, and the number
    of irreducible representation gives us the number of conjugacy classes as well.

    \questionpart
    \textbf{D.} Assume the number of elements in each conjugacy class are $x_n$ for $n \in \{1, 2, 3, 4, 5\}$.
    The character vectors must be orthogonal for the different irreducible representations.
    Disregarding the unknown representation, we get $\binom{4}{2}$ equations. Solving them 
    using a computer, we get $x_5$ as an independent parameter, but it must be a multiple of 6 for
    integer solutions for the rest. So, assuming $x_5$ to be 6, we get $(x_n) = [1, 8, 3, 6, 6]$ and
    we can solve to get $[p, q, r, s, t] = [-3, 0, 1, -1, 1]$ as a reduced solution.

\end{alphaparts}