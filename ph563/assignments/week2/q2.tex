\begin{alphaparts}
    \questionpart
    We can write down the conjugacy classes and the number of elements in each using 
    combinatorial split of 6 elements into the number of required cycles as 
    \begin{center}
        \begin{tabular*}{0.5\textwidth}[ht]{c c}
            Cycle structure     &  Number of elements in class \\ 
                             \hline
            (6, 0, 0, 0, 0, 0)  &  1                           \\
            (4, 1, 0, 0, 0, 0)  &  15                          \\
            (2, 2, 0, 0, 0, 0)  &  45                          \\
            (0, 3, 0, 0, 0, 0)  &  15                          \\
            (3, 0, 1, 0, 0, 0)  &  40                          \\
            (1, 1, 1, 0, 0, 0)  &  120                         \\
            (0, 0, 2, 0, 0, 0)  &  40                          \\
            (2, 0, 0, 1, 0, 0)  &  90                          \\
            (0, 1, 0, 1, 0, 0)  &  90                          \\
            (1, 0, 0, 0, 1, 0)  &  144                         \\
            (0, 0, 0, 0, 0, 1)  &  120                         \\
        
        \end{tabular*}
    \end{center}

    \textbf{Left action (126)(45)(3) $\mid$ Right action (13)(24)(56).} 
    We can calculate it simply by multiplying the individual
    cycles to get a final cycle structure.

    \questionpart
    \begin{itemize}
        \item \textbf{No.} \\
        Take $b = a + 1$. There is a clear commutation violation.
    
        \item \textbf{$i \in \{-1, 0, +1\}$.} \\
        Since $P_a$ is its own inverse, the operation $P_a \cdot \, \cdot P_a$
        is simply the inner operation in a "morphed space" where $P_a$ had acted, and
        then transorming back, the same intuition carried for similarity actions of matrices.
        The equivalence is trivially violated if $P_a$ and $P_{a+i}$ share no common elements, and 
        trivially holds if $i = 0$ as they are the same elements. If $i \not \in \{-1, +1\}$, i.e. 
        in the case where there is one element is common,
        it is easily demostrated this holds considering three elements 
        as in Figure \ref{fig:aiswap}. It is simply a 3 element reversal.
    \end{itemize}

        \begin{figure}
            \centering
            \begin{subfigure}[b]{0.3\textwidth}
                \raggedright
                Start\\
                $$1~~2~~3$$
                $P_1$\\
                $$2~~1~~3$$
                $P_2$\\
                $$2~~3~~1$$
                $P_1$\\
                $$3~~2~~1$$
            \end{subfigure}\hspace{1cm}
            \hspace{3cm}
            \begin{subfigure}[b]{0.3\textwidth}
                \raggedright
                Start\\
                $$1~~2~~3$$
                $P_2$\\
                $$1~~3~~2$$
                $P_1$\\
                $$3~~1~~2$$
                $P_2$\\
                $$3~~2~~1$$
            \end{subfigure}
            \caption{Similarity swapping with a common element.}
            \label{fig:aiswap}
        \end{figure}

    \questionpart
    The stereographic projection is as in Figure \ref{fig:schoen}.

    \begin{figure}[ht]
        \centering
        \scalebox{0.5}{\input{fig/schoen.pdf_tex}}
        \caption{Stereographic Projection / Schoenflies Notation.}
        \label{fig:schoen}
    \end{figure}

    The conjugacy classes will be:
    \begin{itemize}
        \item $\{e\}$
        \item $\{\cyclic_{4}, \cyclic_{4}^{3}\}$
        \item $\{\cyclic_{2}\}$
        \item $\{U_{2}, U_{2}\cyclic_{4}, U_{2}\cyclic_{2}, U_{2}\cyclic_{4}^{3}\}$
        \item $\{\sigma_{d}, \sigma_{d}\cyclic_{4}, \sigma_{d}\cyclic_{2}, \sigma_{d}\cyclic_{4}^{3}\}$
        \item $\{U_{2}\sigma_{d}, U_{2}\sigma_{d}\cyclic_{2}\}$
        \item $\{U_{2}\sigma_{d}\cyclic_{4}, U_{2}\sigma_{d}\cyclic_{4}^{3}\}$
    \end{itemize}

\end{alphaparts}