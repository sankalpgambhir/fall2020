\question*{
    Simple DAC
}

\begin{arabicparts}
    \questionpart To generate a DAC from the circuit:
    \begin{enumerate}
        \item \textbf{2.} A low value and a high value.
        \item \textbf{$R_x, R_x, 2 \cdot R_x$ from the left.} Actual value
                does not matter, unless considering a load connected to this circuit.
                Generally I would use very large resistors and connect a buffer after 
                to minimize power draw.
        \item \textbf{$V_{out} = V_{ref} \cdot \frac{B_0}{2}.$} The resistors simply act as a divider.
        \item \textbf{2.5V.}
        \item \textbf{0V.}
    \end{enumerate}

    \questionpart
    \textbf{$V_{out} = V_{ref} \cdot \left( \frac{B_3}{2} +
            \frac{B_2}{4} + \frac{B_1}{8} + \frac{B_0}{16}\right).$} See \autoref{fig:dac4}
            for the circuit.

            It is a good idea to restrict to only two (proportional) values as multiple independent
            values would generate a much higher error. It is relatively easier to manufacture two
            matched values to a good accuracy.

            The impedances are:
            \begin{itemize}
                \item \textbf{$\frac{128}{43} \cdot R_x$} looking in from $B_0$.
                \item \textbf{$R_x$} looking in from $V_{out}$.
            \end{itemize}
    
    \begin{figure}[ht]
        \centering
        \scalebox{0.75}{\input{fig/dac.pdf_tex}}
        \caption{4-bit DAC design.}
        \label{fig:dac4}
    \end{figure}

\end{arabicparts}