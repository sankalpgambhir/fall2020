\newpage

\question*{Analog Input}

\begin{arabicparts}
    
    \questionpart
    Quite mixed with the next part, so combined answer in \ref{part:c2}
    \questionpart
    \label{part:c2}
    The analog ports A0-A5 were connected to a voltage divider between
    5V and GND taken from the Uno itself as the circuit in Fig \ref{subfig:voltdiv}
    and the program as in \ref{subfig:analogcode}. The actual setup was as already
    included in Fig \ref{fig:arduinoconnected}.

    The wires were moved between the three measuring points during the course 
    of the execution to collect data.

    \begin{figure}[ht]
        \centering
        \begin{subfigure}[b]{0.7\textwidth}
            \def\svgwidth{0.5\textwidth}
            \scalebox{1.1}{\input{fig/voltdiv.pdf_tex}}
            \caption{Voltage divider circuit}
            \label{subfig:voltdiv}
        \end{subfigure}
        %
        \begin{subfigure}[b]{0.7\textwidth}
            \begin{lstlisting}[language=C++]
    void setup(){
        Serial.begin(9600);
    }

    void loop(){
        Serial.print("\nValues : ");
        for(int port = A0; port <= A5; port++){
            Serial.print(analogRead(port));
            Serial.print(' ');
        }
        delay(3000);
    }
            \end{lstlisting}
            \caption{Code used for measuring analog inputs}
            \label{subfig:analogcode}
        \end{subfigure}
        %
        \caption{Setup for the analog input measuring}
        \label{fig:analogsetup}
    \end{figure}

    The output received was as following, ordered as A0 to A5.

    \begin{verbatim}
        ▲ sankalp ~ tail -f /dev/ttyUSB0

        Values : 1023 1023 1023 511 0 0 
        Values : 1023 1023 1023 512 0 0 
        Values : 1023 1023 1023 769 0 0 
        Values : 1023 1023 1023 1023 0 72 
        Values : 1023 1023 1023 1023 512 512 
        Values : 1023 1023 1022 1023 512 512 
        Values : 1023 1023 1023 1023 512 512       
    \end{verbatim}

    It was interesting to see the values in the middle as I managed to read them
    while moving wires around. Not sure if this is dangerous (hopefully not!) but
    it was nice to see the input values changing, presumably due to a buffering
    circuit to smooth over the input in some way?
    
\end{arabicparts}

